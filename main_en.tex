% FORMAT AND PACKAGES
% {
\documentclass[a4paper]{article}
\usepackage{a4wide,amssymb,epsfig,latexsym,multicol,array,hhline,fancyhdr}
\usepackage{vntex}
\usepackage{amsmath}
\usepackage{lastpage}
\usepackage[lined,boxed,commentsnumbered]{algorithm2e}
\usepackage{enumerate}
\usepackage{xcolor}
\usepackage{graphicx}							% Standard graphics package
\usepackage{array}
\usepackage{tabularx, caption}
\usepackage{multirow}
\usepackage{multicol}
\usepackage{rotating}
\usepackage{graphics}
\usepackage{geometry}
\usepackage{setspace}
\usepackage{epsfig}
\usepackage{tikz}
\usepackage{xfrac}
\usepackage{bm}
\usepackage{biblatex}
\usepackage[colorlinks]{hyperref}
% \usepackage[acronym,toc]{glossaries}
% \usepackage[symbols,nogroupskip,nonumberlist]{glossaries-extra}
\usepackage[
 sort=none,% no sorting or indexing required
 abbreviations,% create list of abbreviations
 symbols,% create list of symbols
 stylemods,style=list, % set the default glossary style
 nogroupskip, nonumberlist, nomain
]{glossaries-extra}


% FORMATTING
% {
\DeclareMathOperator{\arccot}{arccot}
\captionsetup[table]{name=Table}
\captionsetup[figure]{name=Figure}
\newenvironment{Description}{\list{}{%
    \let\makelabel\descriptionlabel    % this comes from the original description environment
    \setlength{\rightmargin}{\leftmargin}% this comes from the original quote environment
    \setlength{\labelwidth}{0pt}%          this is new
    }}{\endlist}

\addbibresource{citations.bib}
    
\hypersetup{urlcolor=blue,linkcolor=black,citecolor=black,colorlinks=true} 
\usetikzlibrary{arrows,snakes,backgrounds}
\definecolor{mathblue}{RGB}{0,114,188}
% \makeatletter  \def\m@th{\mathsurround\z@\color{mathblue}} \makeatother
% \everymath{\color{mathblue}}
% \setmathfont[Color=000000]{Arial}
%\usepackage{pstcol} 								% PSTricks with the standard color package
\newtheorem{theorem}{{\bf Theorem}}
\newtheorem{property}{{\bf Property}}
\newtheorem{proposition}{{\bf Proposition}}
\newtheorem{corollary}[proposition]{{\bf Corollary}}
\newtheorem{lemma}[proposition]{{\bf Lemma}}

\AtBeginDocument{\renewcommand{\listfigurename}{List of Figures}}
\AtBeginDocument{\renewcommand{\listtablename}{List of Tables}}
\AtBeginDocument{\renewcommand*\contentsname{Contents}}
\AtBeginDocument{\renewcommand*\refname{References}}
%\usepackage{fancyhdr}

\setlength{\headheight}{40pt}
\pagestyle{fancy}
\fancyhead{} % clear all header fields
\fancyhead[L]{
 \begin{tabular}{rl}
    \begin{picture}(25,15)(0,0)
    \put(0,-8){\includegraphics[width=8mm, height=8mm]{image/hcmut.png}}
    %\put(0,-8){\epsfig{width=10mm,figure=hcmut.eps}}
   \end{picture}&
	%\includegraphics[width=8mm, height=8mm]{hcmut.png} & %
	\begin{tabular}{l}
		\textbf{\bf \ttfamily University of Technology, Ho Chi Minh City}\\
		\textbf{\bf \ttfamily Faculty of Computer Science and Engineering}
	\end{tabular} 	
 \end{tabular}
}
\fancyhead[R]{
	\begin{tabular}{l}
		\tiny \bf \\
		\tiny \bf 
	\end{tabular}  }
\fancyfoot{} % clear all footer fields
\fancyfoot[L]{\scriptsize \ttfamily Programming Intergration Project - Academic year 2025 - 2026}
\fancyfoot[R]{\scriptsize \ttfamily Page {\thepage}/\pageref{LastPage}}
\renewcommand{\headrulewidth}{0.3pt}
\renewcommand{\footrulewidth}{0.3pt}

\setcounter{secnumdepth}{4}
\setcounter{tocdepth}{4}

\makeatletter
\newcounter {subsubsubsection}[subsubsection]
\renewcommand\thesubsubsubsection{\thesubsubsection .\@alph\c@subsubsubsection}
\newcommand\subsubsubsection{\@startsection{subsubsubsection}{4}{\z@}%
                                     {-3.25ex\@plus -1ex \@minus -.2ex}%
                                     {1.5ex \@plus .2ex}%
                                     {\normalfont\normalsize\bfseries}}
\newcommand*\l@subsubsubsection{\@dottedtocline{3}{10.0em}{4.1em}}
\newcommand*{\subsubsubsectionmark}[1]{}
% \def\m@th{\mathsurround\z@\color{mathblue}}
\makeatother
% }
% }

% ACRONYMS & SYMBOLS
% {
% \makeglossaries
\setabbreviationstyle{long-short}
\newabbreviation{ode}{ODE}{(First-Order) Ordinary Differential Equation}
\newabbreviation{ivp}{IVP}{Initial-Value Problem}
\newabbreviation{lte}{LTE}{Local Truncation Error}
\newabbreviation{ds}{DS}{Dynamical System}
\newabbreviation{fig}{Fig.}{Figure}
\newabbreviation{tab}{Tab.}{Table}
\newabbreviation{sys}{Sys.}{System of Equations}
\newabbreviation{eq}{Eq.}{Equation}
\newabbreviation{eg}{e.g.}{For Example}
\newabbreviation{ie}{i.e.}{That Is}
% \glsnoexpandfields
\glsxtrnewsymbol[description = {Set of natural numbers}]{natural}{\ensuremath{\mathbb{N}}}
\glsxtrnewsymbol[description = {Set of real numbers}]{real}{\ensuremath{\mathbb{R}}}
\glsxtrnewsymbol[description = {Set of positive real numbers}]{real_positive}{\ensuremath{\mathbb{R}^+}}

% }

% Code listing
\usepackage{listings}
% \usepackage{xcolor}

\definecolor{codegreen}{rgb}{0,0.6,0}
\definecolor{codegray}{rgb}{0.5,0.5,0.5}
\definecolor{codepurple}{rgb}{0.58,0,0.82}
\definecolor{backcolour}{rgb}{0.95,0.95,0.92}

\lstdefinestyle{mystyle}{
    backgroundcolor=\color{backcolour},   
    commentstyle=\color{codegreen},
    keywordstyle=\color{magenta},
    numberstyle=\tiny\color{codegray},
    stringstyle=\color{codepurple},
    basicstyle=\ttfamily\footnotesize,
    breakatwhitespace=false,         
    breaklines=true,                 
    captionpos=b,                    
    keepspaces=true,                 
    numbers=left,                    
    numbersep=5pt,                  
    showspaces=false,                
    showstringspaces=false,
    showtabs=false,                  
    tabsize=2
}

\lstset{style=mystyle}
% 

% Float [H]
\usepackage{float}
%

% DOCUMENT
\begin{document}

% TITLE PAGE
\begin{titlepage}
\begin{center}
VIETNAM NATIONAL UNIVERSITY, HO CHI MINH CITY \\
UNIVERSITY OF TECHNOLOGY \\
FACULTY OF COMPUTER SCIENCE AND ENGINEERING
\end{center}

\vspace{1cm}

\begin{figure}[h!]
\begin{center}
\includegraphics[width=3cm]{image/hcmut.png}
\end{center}
\end{figure}

\vspace{1cm}


\begin{center}
\begin{tabular}{c}
\multicolumn{1}{c}{\textbf{{\Large Programming Intergration Project (CO3101)}}}\\
~~\\
\hline
\\
\multicolumn{1}{l}{\textbf{{\Large Group 4:}}}\\

\parbox{0.9\textwidth}{\centering\textbf{\textit{{\LARGE  ``Research and build AI chatbots using Retrieval-Augmented Generation for a music-related website, with Speech-to-Text and Text-to-Speech integration''}}}}\\
\\
\hline
\end{tabular}
\end{center}

\vspace{2cm}

\begin{table}[h]
\centering
    \begin{tabular}{rl}
    \hspace{3 cm}\textbf{Instructor(s)}:
    & Nguyễn Quốc Minh\\

    & \\[10pt]
    \textbf{Students}: &  Nguyễn Thiện Minh - 2312097  \\
    &  Huỳnh Đức Nhân - 2312420 \\
    &  Phạm Trần Minh Trí - 2313622 \\
    % &  Bùi Tiến Dũng - 2113055 \emph{(Group L04 - Team 02)}\\
    \end{tabular}
\end{table}

\begin{center}
{\footnotesize HO CHI MINH CITY, DECEMBER 2025}
\end{center}
\end{titlepage}

\pagebreak
\tableofcontents
\pagebreak

% Glossaries
% {}
% \printunsrtglossary[type={symbols}, title={List of Symbols}]
% \printunsrtglossary[type={abbreviations}, title={List of Acronyms}]
\pagebreak
\listoffigures
\listoftables
\pagebreak
\addcontentsline{toc}{section}{\listfigurename}
\addcontentsline{toc}{section}{\listtablename}

% }

% Member list
\section*{Member list \& Workload}
\addcontentsline{toc}{section}{Member list \& Workload}
\begin{center}
\begin{table}[h]
\centering
\begin{tabular}{|c|c|c|l|c|}
\hline
\textbf{No.} & \textbf{Fullname} & \textbf{Student ID} & \textbf{Problems} & \textbf{\% done}\\
\hline 
%%%%%Student 1%%%%%%%%%%
\multirow{3}{*}{1} & \multirow{3}{*}{Nguyễn Thiện Minh} & \multirow{3}{*}{2312097} & 
- Exercise 1: 1.2& \multirow{3}{*}{100\%}\\
 & &  & - Exercise 2&\\
 & &  & - Exercise 3: 3.2&\\
\hline 
%%%%%Student 2%%%%%%%%%%%
\multirow{4}{*}{2} & \multirow{4}{*}{Huỳnh Đức Nhân} & \multirow{4}{*}{2312420} & 
- Exercise 1: 1.3& \multirow{4}{*}{100\%}\\
 & &  & - Exercise 2&\\
 & &  & - Exercise 3: 3.1&\\
 & &  & - Exercise 4&\\
\hline
%%%%%Student 3%%%%%%%%%%%
\multirow{3}{*}{3} & \multirow{3}{*}{Phạm Trần Minh Trí} & \multirow{3}{*}{2313622} & 
- Exercise 1: 1.1& \multirow{3}{*}{100\%}\\
 & &  & - Exercise 4&\\
 & &  & - \LaTeX  &\\
\hline
\end{tabular}
\caption{\label{table1}Member list \& workload}
\end{table}
\end{center}


\pagebreak


% \section{Section 1}
% \subsection{Subsubsection 1}

% \begin{figure}[H]
%     \centering
%     \includegraphics[width=0.5\linewidth]{image/hcmut.png}
%     \caption{Image example}
%     \label{fig:enter-label}
% \end{figure}

% \begin{lstlisting}[language=Python, caption=Python example]
% import numpy as np
    
% def incmatrix(genl1,genl2):
%     m = len(genl1)
%     n = len(genl2)
%     M = None #to become the incidence matrix
%     VT = np.zeros((n*m,1), int)  #dummy variable
    
%     #compute the bitwise xor matrix
%     M1 = bitxormatrix(genl1)
%     M2 = np.triu(bitxormatrix(genl2),1) 

%     for i in range(m-1):
%         for j in range(i+1, m):
%             [r,c] = np.where(M2 == M1[i,j])
%             for k in range(len(r)):
%                 VT[(i)*n + r[k]] = 1;
%                 VT[(i)*n + c[k]] = 1;
%                 VT[(j)*n + r[k]] = 1;
%                 VT[(j)*n + c[k]] = 1;
                
%                 if M is None:
%                     M = np.copy(VT)
%                 else:
%                     M = np.concatenate((M, VT), 1)
                
%                 VT = np.zeros((n*m,1), int)
    
%     return M
% \end{lstlisting}


\section{Introduction}

\subsection{Background and Motivation}

In recent years, the field of Artificial Intelligence has witnessed a paradigm shift with the emergence of Large Language Models (LLMs) such as GPT-4, Gemini, and Llama. These models, built upon the Transformer architecture, have demonstrated remarkable capabilities in natural language understanding, generation, and reasoning. They can draft emails, write code, and even engage in complex conversations that mimic human interaction.

Despite their impressive performance, LLMs face inherent limitations. One significant challenge is "hallucination," where models generate plausible but factually incorrect information. Additionally, purely parametric models are limited by their training data cut-off dates and lack access to private or real-time proprietary data. For specialized domains, such as a dedicated music platform, generic LLMs may not provide specific, deep, or curated knowledge required to satisfy user queries accurately.

\subsection{Problem Statement}

To address these limitations, Retrieval-Augmented Generation (RAG) has emerged as a powerful framework. By grounding the generation process in external, verifiable knowledge sources, RAG significantly reduces hallucinations and allows models to access up-to-date information without the need for frequent retraining.

Furthermore, user interaction with AI systems is evolving beyond text-based interfaces. The integration of Speech-to-Text (STT) and Text-to-Speech (TTS) technologies enables more natural, hands-free, and accessible communication. Building a comprehensive AI system that combines advanced retrieval mechanisms with multi-modal interaction capabilities represents a significant engineering challenge that requires integrating various state-of-the-art components.

% \subsection{Project Scope and Objectives}

% This project, titled \textbf{"Research and build AI chatbots using Retrieval-Augmented Generation for a music-related website, with Speech-to-Text and Text-to-Speech integration"}, aims to develop a sophisticated conversational agent tailored for the music domain. The primary objectives of this project are:

% \begin{enumerate}
%     \item To research and understand the theoretical foundations of modern AI, including ANNs, Transformers, and Agents.
%     \item To investigate the RAG framework, ranging from Naive implementations to advanced techniques like Reranking and Reasoning.
%     \item To implement a functional AI Agent using LangChain that can retrieve information from a curated music knowledge base.
%     \item To enhance user experience by integrating Speech-to-Text and Text-to-Speech capabilities for voice-based interaction.
%     \item To evaluate the system's performance in terms of retrieval accuracy and response relevance.
% \end{enumerate}

\subsection{Report Structure}

The remainder of this report is organized as follows:

\begin{itemize}
    \item \textbf{Section 2} provides the theoretical prerequisites, covering Artificial Neural Networks, Transformer architectures, and Large Language Models.
    \item \textbf{Section 3} introduces the concepts of AI Agents, Universal RAG (URAG), and the LangChain framework.
    \item \textbf{Section 4} presents a survey of Retrieval-Augmented Generation, detailing the standard "Retrieve-Read" pipeline.
    \item \textbf{Section 5} discusses advanced RAG techniques, such as Reranking, to improve retrieval quality and reduce noise.
    \item \textbf{Section 6} details the implementation of our music chatbot, utilizing LangChain, Wikipedia data loaders, and voice technologies.
    \item \textbf{Section 7} presents the evaluation methodology and results, using metrics like Cosine Similarity and LLM-as-a-Judge.
    \item \textbf{Section 8} concludes the report and discusses potential future improvements.
\end{itemize}

\pagebreak
\section{Prerequisite: ANN, Transformer, LLM}
\pagebreak
\section{Agent, URAG, LangChain}

\subsection{AI Agent}

An AI agent is a system that leverages an artificial intelligence model to interact with its environment in order to achieve a user-defined objective. Unlike traditional AI systems that produce static outputs, an AI agent integrates reasoning, planning, and action execution to complete tasks in a dynamic and autonomous manner.

\subsubsection{Introduction to AI Agents}

An AI agent operates by receiving a user request, reasoning about the task, selecting appropriate tools, and executing actions to fulfill the objective. This process allows agents to solve multi-step problems that require interaction with external systems and continuous adaptation based on feedback.

\begin{figure}[H]
    \centering
    \includegraphics[width=0.9\textwidth]{image/agent_workflow.png}
    \caption{High-level AI agent workflow: user request, reasoning, and tool-based action}
    \label{fig:agent_workflow}
\end{figure}

\subsubsection{Core Components of an AI Agent}

AI agents consist of two main components:

\begin{itemize}
    \item \textbf{The Brain (AI Model)}:  
    This component is typically a Large Language Model (LLM) that performs reasoning, understands context, plans future steps, and decides which actions or tools should be used.
    
    \item \textbf{The Body (Capabilities and Tools)}:  
    This component enables the agent to act within its environment. It includes external tools and APIs such as search engines, code execution environments, and databases, as well as memory and observation mechanisms.
\end{itemize}

\subsubsection{Levels of Agent Agency}

AI agents can be categorized according to their level of autonomy, commonly referred to as their \textit{agency level}. As the agency level increases, the agent gains greater control over program execution and decision-making.

\subsubsection{Thought--Action--Observation Cycle}

The behavior of an AI agent follows a continuous loop known as the \textit{Thought--Action--Observation} cycle. In this cycle, the agent first reasons about the problem, then performs an action, and finally observes the result of that action. The observation is used to update the agent’s internal state and guide subsequent reasoning steps.

\begin{figure}[H]
    \centering
    \includegraphics[width=0.8\textwidth]{image/tao_cycle.png}
    \caption{The Thought--Action--Observation (TAO) cycle in an AI agent}
    \label{fig:tao_cycle}
\end{figure}

This iterative cycle enables agents to adapt their behavior dynamically based on environmental feedback.

\subsubsection{Practical Illustration of the TAO Cycle}

A common example of the Thought--Action--Observation cycle is a task that requires external information, such as retrieving real-time weather data. The agent reasons about the user’s request, calls an appropriate external API, observes the returned data, and then produces a response.

\begin{figure}[H]
    \centering
    \includegraphics[width=0.9\textwidth]{image/tao_weather_step1.png}
    \caption{Reasoning and action steps using an external weather API}
    \label{fig:tao_weather_1}
\end{figure}

\subsubsection{Observation and Response Generation}

After executing an action, the agent observes the outcome and integrates the feedback into its internal context. Based on this updated information, the agent performs additional reasoning to generate the final response for the user.

\begin{figure}[H]
    \centering
    \includegraphics[width=0.85\textwidth]{image/tao_weather_step2.png}
    \caption{Observation feedback and final response generation}
    \label{fig:tao_weather_2}
\end{figure}

\subsubsection{Internal Reasoning: Chain-of-Thought and ReAct}

Reasoning is a critical capability of AI agents. Two widely used prompting approaches are:

\begin{itemize}
    \item \textbf{Chain-of-Thought (CoT)}:  
    A technique that encourages the model to reason step-by-step before generating a final answer.
    
    \item \textbf{ReAct (Reasoning and Acting)}:  
    A technique that interleaves reasoning steps with actions, allowing the agent to think, act using tools, observe results, and continue reasoning until the task is completed.
\end{itemize}

\subsubsection{Actions: Enabling Interaction with the Environment}

Actions allow an AI agent to engage with its environment and accomplish tasks. Common categories of actions include information gathering, tool usage, environment interaction, and communication.

\subsubsection{Observation and Adaptation}

After executing an action, the agent observes the outcome and integrates the feedback into its internal context. This enables the agent to update its memory, refine its strategy, and improve performance in future interactions.




\subsection{Unified Hybrid RAG}

\subsection{Introduction to LangChain}

% \subsection{LangChain Framework}

LangChain is a comprehensive framework designed to support the development, deployment, and monitoring of applications powered by Large Language Models (LLMs). It provides modular components and abstractions that simplify the construction of complex LLM-based systems such as Retrieval-Augmented Generation (RAG) pipelines and agentic workflows.

LangChain supports the full lifecycle of an LLM application. During the {development} phase, developers can build applications using LangChain’s core components, including prompt templates, chains, retrievers, vector stores, and integrations with third-party tools and APIs. For more advanced agentic behaviors, LangChain introduces {LangGraph}, which enables the definition of multi-step agents with explicit states, nodes, and control flows. This graph-based design allows agents to reason, invoke tools, and iterate over multiple steps in a structured and controllable manner.

In the {production} phase, LangChain is complemented by {LangSmith}, a platform that provides observability, debugging, and evaluation capabilities. LangSmith allows developers to inspect prompt execution, track intermediate steps, monitor latency and costs, and systematically evaluate model outputs. These features are particularly important for RAG systems, where both retrieval quality and generation accuracy must be continuously assessed.

For {deployment}, LangChain offers the LangGraph Platform, which facilitates the deployment and scaling of agentic workflows. This platform-oriented approach makes LangChain suitable not only for experimentation but also for real-world applications.

Within a RAG architecture, LangChain structures the workflow into two main stages. The first stage is {indexing}, an offline process that involves loading documents, splitting them into chunks, generating embeddings using an embedding model, and storing these embeddings in a vector database. The second stage is {retrieval and generation}, which occurs at runtime: relevant document chunks are retrieved from the vector store based on the user query, combined with the query into a structured prompt, and passed to an LLM to generate a final response.

\begin{figure}[H]
    \centering
    \includegraphics[width=0.75\linewidth]{image/3_01.png}
    \caption{RAG - Indexing}
    % \label{fig:enter-label}
\end{figure}



LangChain provides a flexible and extensible foundation for building RAG-based and agent-driven applications, enabling developers to integrate LLMs, retrieval systems, and external tools into a unified and production-ready framework.

\begin{figure}[H]
    \centering
    \includegraphics[width=0.75\linewidth]{image/3_02.png}
    \caption{RAG - Retrieval and Generation}
    % \label{fig:enter-label}
\end{figure}



\pagebreak
\section{Retrieval-Augmented Generation for Large Language Models: A Survey}

Link to the paper: \url{https://arxiv.org/abs/2312.10997} \cite{gao2023retrieval}

\subsection{Overview of RAG}

Retrieval-Augmented Generation (RAG) is a paradigm that enhances large language models (LLMs) by incorporating external knowledge retrieved at inference time. Instead of relying solely on parametric knowledge stored in model weights, RAG dynamically retrieves relevant documents and uses them as additional context for response generation.

\begin{figure}[H]
    \centering
    \includegraphics[width=0.75\linewidth]{image/4_01.png}
    \caption{RAG overview}
    % \label{fig:enter-label}
\end{figure}


\subsubsection{Naive RAG}

The Naive RAG framework follows a simple \textit{Retrieve--Read} pipeline consisting of three main stages: indexing, retrieval, and generation. During indexing, raw data sources such as PDFs, HTML pages, or Word documents are converted into plain text, segmented into smaller chunks, and encoded into vector representations stored in a vector database. In the retrieval stage, the user query is embedded and compared with stored vectors using similarity metrics to obtain the top-$K$ most relevant chunks. Finally, in the generation stage, the LLM produces an answer based on the user query and the retrieved context, optionally incorporating conversation history.

Despite its simplicity, Naive RAG suffers from several limitations. Retrieval may lack precision and recall, resulting in irrelevant or missing information. During generation, the model may hallucinate content not supported by the retrieved context or produce outputs with bias or irrelevance. Moreover, effectively integrating retrieved information across different tasks remains challenging, often leading to redundant, incoherent, or overly extractive responses.

\subsubsection{Advanced RAG}

Advanced RAG aims to improve retrieval quality and context utilization through enhanced pre-retrieval and post-retrieval techniques. Pre-retrieval optimization focuses on improving indexing structures and query formulation, including data granularity control, metadata alignment, mixed retrieval strategies, and query rewriting or expansion. Post-retrieval optimization emphasizes effective context integration, such as re-ranking retrieved documents to prioritize relevance and compressing context to reduce noise and prompt length.

\subsubsection{Modular RAG}

Modular RAG extends beyond fixed retrieval-generation pipelines by introducing specialized, interchangeable modules. These include search modules for heterogeneous data sources, RAG-Fusion for multi-query expansion and re-ranking, and memory modules that maintain a continuously updated retrieval memory pool. Additional components such as routing, prediction, and task adapters allow RAG systems to dynamically select retrieval pathways and adapt to downstream tasks.

This modular design enables flexible retrieval patterns, including Rewrite--Retrieve--Read, Generate--Read, and hybrid retrieval strategies that combine keyword-based, semantic, and vector-based search. As a result, Modular RAG exhibits strong adaptability and scalability across diverse applications.

\begin{figure}[H]
    \centering
    \includegraphics[width=0.75\linewidth]{image/4_02.png}
    \caption{Types of RAG}
    % \label{fig:enter-label}
\end{figure}

\subsection{Retrieval}

Retrieval is a core component of RAG, responsible for identifying and supplying relevant external knowledge to the generation model. The effectiveness of a RAG system heavily depends on retrieval source selection, indexing strategies, query optimization, and embedding quality.

\subsubsection{Retrieval Sources and Granularity}

Retrieval sources can be categorized into unstructured, semi-structured, and structured data. Unstructured text data, such as Wikipedia articles or domain-specific documents, is the most common source. Semi-structured data, including PDFs with tables, presents challenges due to structural complexity, while structured sources like knowledge graphs offer precise and verified information at the cost of higher construction and maintenance effort.

Retrieval granularity ranges from tokens and sentences to chunks and full documents. Coarse-grained retrieval provides richer context but may introduce redundancy and noise, whereas fine-grained retrieval improves precision but risks losing essential semantic information.

\subsubsection{Indexing Optimization}

Indexing optimization techniques aim to balance context richness and efficiency. Chunking strategies play a critical role, where large chunks capture broader context but increase noise and computational cost, while small chunks reduce noise but may lack sufficient information. The \textit{Small-to-Big} approach mitigates this trade-off by retrieving smaller units and expanding context hierarchically.

Metadata attachments, such as page numbers or timestamps, enable filtered retrieval and scoped search. Structural indexing methods, including hierarchical document structures and knowledge graph indices, further enhance retrieval speed and relevance. Techniques like Reverse HyDE leverage LLMs to generate potential questions that each chunk can answer, improving retrievability.

\subsubsection{Query Optimization}

Query optimization improves retrieval effectiveness by refining or expanding user queries. Query expansion and multi-query techniques enrich the query with additional context, while sub-query decomposition breaks complex questions into simpler ones. Query transformation methods include rewriting queries, generating hypothetical answers (HyDE), and step-back prompting to retrieve higher-level contextual information.

Query routing mechanisms further enhance retrieval by directing queries to appropriate data sources or pipelines using metadata-based or semantic routing strategies.

\subsubsection{Embeddings and Adapters}

Modern RAG systems often employ hybrid retrieval that combines sparse retrievers, such as BM25 for keyword matching, with dense retrievers based on neural embeddings for semantic understanding. Embedding models can be fine-tuned for domain-specific tasks, with LM-supervised retrievers aligning retrieval objectives with generation outcomes using LLM feedback.

When fine-tuning is impractical, adapter-based methods provide lightweight alternatives. These include prompt retrievers, bridging modules that transform retrieved content into LLM-friendly formats, and plug-in knowledge generators that replace or augment traditional retrievers in white-box settings.

\subsection{Generation}

Generation is the final stage in a Retrieval-Augmented Generation (RAG) pipeline, where the Large Language Model (LLM) produces the final response based on the user query and the augmented context retrieved from external knowledge sources. This stage directly determines the quality, coherence, and usefulness of the system’s output.

\subsubsection{Overview of the Generation Stage}

In the basic RAG setting, also referred to as \textit{Naive RAG}, the generation stage receives the user query together with the retrieved documents as input. These elements are combined into a single prompt, which is then passed to a frozen LLM to generate the final answer.

\begin{figure}[H]
    \centering
    \includegraphics[width=0.3\textwidth]{image/generation_naive_rag.png}
    \caption{Generation stage in Naive RAG, where retrieved context is directly passed to a frozen LLM via a prompt}
    \label{fig:generation_naive}
\end{figure}

The generated response is expected to be fluent, context-grounded, and aligned with the retrieved evidence. However, this approach may suffer from noisy or redundant context, which can degrade answer quality.

\subsubsection{Context Curation for Improved Generation}

To address the limitations of Naive RAG, advanced RAG systems introduce a \textit{context curation} step before generation. The goal of context curation is to improve relevance and reduce noise in the retrieved context supplied to the LLM.

Key components of context curation include:
\begin{itemize}
    \item \textbf{Reranking}: reorders retrieved documents to prioritize the most relevant ones.
    \item \textbf{Context Selection or Compression}: filters, summarizes, or shortens retrieved content to avoid overly long prompts.
\end{itemize}

\begin{figure}[H]
    \centering
    \includegraphics[width=0.3\textwidth]{image/generation_context_curation.png}
    \caption{Advanced RAG with post-retrieval context curation before generation}
    \label{fig:generation_context}
\end{figure}

By providing a more concise and focused augmented prompt, context curation enables the LLM to generate higher-quality and more accurate responses.

\subsubsection{LLM Fine-tuning for Generation}

Beyond improving the quality of the input context, generation performance can be further enhanced by fine-tuning the LLM itself. The objective of LLM fine-tuning is to adapt the model to domain-specific knowledge and desired response styles.

The key benefits of LLM fine-tuning include:
\begin{itemize}
    \item \textbf{Domain Adaptation}: improves understanding and reasoning in specialized domains.
    \item \textbf{Output Alignment}: aligns tone, structure, and formatting with predefined guidelines.
    \item \textbf{Reduced Hallucination}: encourages stronger reliance on retrieved evidence.
\end{itemize}

\begin{figure}[H]
    \centering
    \includegraphics[width=0.3\textwidth]{image/generation_llm_finetuning.png}
    \caption{Generation stage using a fine-tuned LLM in an Advanced RAG pipeline}
    \label{fig:generation_finetune}
\end{figure}

As a result, fine-tuned generation produces responses that are more accurate, coherent, and context-grounded, making it suitable for domain-specific and high-stakes applications.

\subsection{Augmentation}

Augmentation is a key process in Retrieval-Augmented Generation (RAG) systems that enhances answer quality by iteratively refining retrieval and generation. Instead of performing retrieval only once, the system can retrieve additional information, reformulate queries, or adjust context dynamically during the answering process.

\subsubsection{Overview of the Augmentation Process}

In the augmentation process, retrieval and generation are tightly coupled in an iterative loop. After each generation step, the system evaluates whether the current information is sufficient or if additional retrieval is required before producing the final response.

\begin{figure}[H]
    \centering
    \includegraphics[width=0.3\textwidth]{image/augmentation_overview.png}
    \caption{Overview of the iterative augmentation process in RAG}
    \label{fig:augmentation_overview}
\end{figure}

The main purposes of augmentation are:
\begin{itemize}
    \item Handling complex or multi-step reasoning questions.
    \item Producing more accurate and context-grounded answers.
\end{itemize}

\subsubsection{Iterative Retrieval}

Iterative retrieval repeatedly alternates between retrieving new context and generating partial answers. Each generation step provides new signals that guide the next retrieval, gradually enriching the available context.

\begin{figure}[H]
    \centering
    \includegraphics[width=0.3\textwidth]{image/augmentation_iterative.png}
    \caption{Iterative retrieval with repeated retrieval--generation cycles}
    \label{fig:augmentation_iterative}
\end{figure}

The key ideas behind iterative retrieval include:
\begin{itemize}
    \item Retrieval is performed again based on what has been generated so far.
    \item Context is progressively enriched to support step-by-step reasoning.
\end{itemize}

\textbf{Pros and Cons:}
\begin{itemize}
    \item \textbf{Advantages}: more comprehensive and targeted information.
    \item \textbf{Disadvantages}: risk of semantic drift or accumulation of irrelevant context.
\end{itemize}

\subsubsection{Recursive Retrieval}

Recursive retrieval refines the query step-by-step using feedback from previous retrieval results. Each iteration clarifies what information is still missing, leading to progressively more relevant context.

\begin{figure}[H]
    \centering
    \includegraphics[width=0.3\textwidth]{image/augmentation_recursive.png}
    \caption{Recursive retrieval with query transformation and decomposition}
    \label{fig:augmentation_recursive}
\end{figure}

Key characteristics of recursive retrieval include:
\begin{itemize}
    \item Construction of intermediate reasoning structures such as chain-of-thought or clarification trees.
    \item Continuous reformulation of the query to target more specific information.
\end{itemize}

\textbf{Pros and Cons:}
\begin{itemize}
    \item \textbf{Advantages}: improved accuracy and relevance over multiple iterations.
    \item \textbf{Disadvantages}: increased latency due to multiple refinement steps.
\end{itemize}

\subsubsection{Adaptive Retrieval}

Adaptive retrieval allows the LLM to dynamically decide when and what to retrieve during generation. Instead of enforcing retrieval at fixed stages, the model monitors its own confidence and triggers retrieval only when additional context is required.

\begin{figure}[H]
    \centering
    \includegraphics[width=0.3\textwidth]{image/augmentation_adaptive.png}
    \caption{Adaptive retrieval where the LLM actively controls retrieval decisions}
    \label{fig:augmentation_adaptive}
\end{figure}

The key ideas of adaptive retrieval are:
\begin{itemize}
    \item Retrieval is performed on demand rather than at predefined stages.
    \item The LLM behaves like an agent, evaluating intermediate outputs and deciding whether to retrieve more context or proceed to the final answer.
\end{itemize}

\subsubsection{Summary}

In summary, augmentation enhances the RAG pipeline by enabling iterative and adaptive interaction between retrieval and generation. Iterative, recursive, and adaptive retrieval strategies provide increasing levels of flexibility and reasoning capability, allowing RAG systems to better handle complex queries and produce more accurate, context-aware responses.

\subsection{Tasks and Evaluation Framework}

\subsubsection{Downstream Tasks and Benchmarks}
The primary objective of RAG systems is to support various Natural Language Processing (NLP) tasks by providing external context. While Question Answering (QA) remains the most prominent application, the scope of RAG is rapidly expanding.

\begin{itemize}
    \item \textbf{Question Answering (QA):} This is the core application, encompassing several complexities:
    \begin{itemize}
        \item \textit{Single-hop QA:} Requires retrieving information from a single document (e.g., \textbf{Natural Questions (NQ), TriviaQA, SQuAD}).
        \item \textit{Multi-hop QA:} Requires synthesizing information across multiple documents to answer complex queries (e.g., \textbf{HotpotQA, 2WikiMultiHopQA, MuSiQue}).
        \item \textit{Domain-specific QA:} Tailored for specialized fields like medicine or research (e.g., \textbf{Qasper, COVID-QA, MMCU Medical}).
    \end{itemize}
    \item \textbf{Expanded Tasks:} Beyond QA, RAG is utilized in \textbf{Information Extraction (IE)}, \textbf{Dialogue Generation}, and \textbf{Code Search}, where external documentation is vital for accuracy.
\end{itemize}

\subsubsection{Evolution of Evaluation Targets}
Traditional evaluation relied on lexical overlap metrics, but modern RAG research shifts toward component-level analysis.

\begin{enumerate}
    \item \textbf{Traditional Metrics:} Metrics such as \textbf{EM (Exact Match)}, \textbf{F1 score}, and \textbf{ROUGE} were standard. However, they fail to evaluate the internal reasoning of the model or the quality of the retrieved context.
    \item \textbf{Modern RAG Metrics:} Evaluation is now bifurcated into two primary targets:
    \begin{itemize}
        \item \textbf{Retrieval Quality:} Measured by \textbf{Hit Rate}, \textbf{Mean Reciprocal Rank (MRR)}, and \textbf{Normalized Discounted Cumulative Gain (NDCG)}.
        \item \textbf{Generation Quality:} Focused on \textbf{Faithfulness} (factuality), \textbf{Relevance} (answering the intent), and \textbf{Non-harmfulness}.
    \end{itemize}
\end{enumerate}

\subsubsection{Evaluation Aspects: The "RAG Triad" and Key Abilities}
Current research emphasizes three quality scores (often referred to as the RAG Triad) and four essential robustness abilities.

\textbf{The Three Quality Scores:}
\begin{itemize}
    \item \textbf{Context Relevance:} Evaluates whether the retrieved snippets are precise and specific to the query, minimizing "noise" or extraneous content.
    \item \textbf{Answer Faithfulness:} Ensures the generated output is derived strictly from the retrieved context without "hallucinating" external or contradictory information.
    \item \textbf{Answer Relevance:} Measures how well the response directly addresses the user's question.
\end{itemize}

\textbf{Four Essential Abilities:}
\begin{itemize}
    \item \textbf{Information Integration:} The capacity to synthesize insights from multiple disparate documents.
    \item \textbf{Noise Robustness:} The ability to filter out irrelevant or misleading information within the retrieved set.
    \item \textbf{Counterfactual Robustness:} The ability to recognize and disregard known inaccuracies or "fake news" injected into documents.
    \item \textbf{Negative Rejection:} The system's ability to decline answering when the provided documents do not contain the necessary knowledge.
\end{itemize}

\subsection{Discussion}

\begin{itemize}
    \item \textbf{RAG vs. Long-Context LLMs}: With modern LLMs supporting contexts over 200,000 tokens, the necessity of RAG is often debated. However, RAG remains superior due to:
    \begin{itemize}
        \item \textbf{Efficiency:} Processing massive contexts per request is computationally expensive and slow. RAG's chunk-based retrieval is significantly faster and more cost-effective.
        \item \textbf{Observability:} RAG provides clear citations and references, whereas Long-Context generation operates as a "black box," making verification difficult.
    \end{itemize}

    \item \textbf{The Noise Paradox}: Recent studies have identified a "Robustness Paradox": in certain scenarios, including a small amount of irrelevant (noisy) documents can unexpectedly improve model accuracy by over 30\%, suggesting that some noise may act as a regularizer during the reasoning process.

    \item \textbf{Hybrid Approaches}: The most effective current strategy is a hybrid model: using \textbf{Fine-tuning} to adapt the model’s behavior and style, while using \textbf{RAG} to provide updated, dynamic knowledge.

    \item \textbf{Scaling Laws}: While LLMs follow predictable scaling laws, it is unclear if these apply to RAG. Some research suggests an \textbf{"Inverse Scaling Law"}, where smaller, specialized models may outperform larger general-purpose models when equipped with high-quality RAG pipelines.

\end{itemize}
\pagebreak
\section{Reranking, RAG-Reasoning, RAG-RL}
\pagebreak
\section{Implementation: LangChain, Text-to-speech, Speech-to-text}
\pagebreak
\section{Evaluation}

We need to evaluate how effective our RAG system is. We will be rating the RAG feature, not the web search tool here. As discussed in the previous section, the RAG vector store is taken from 76 Wikipedia articles about Music. We will also measure the time it take to retrieve the relevance context from the database.

We use 2 type of evaluation score:

\begin{enumerate}
    \item \textbf{Cosine Similarity (0-1)}: Semantic similarity between query embeddings and document embeddings
    \item \textbf{LLM-as-a-Judge (1-5 scale)}:
    \begin{itemize}
        \item 5 = Highly relevant, directly answers query
        \item 4 = Relevant, provides useful information
        \item 3 = Somewhat relevant, tangentially related
        \item 2 = Minimally relevant
        \item 1 = Not relevant
    \end{itemize}
\end{enumerate}

We use 10 text querries: Basic factual queries (1-2), Temporal queries (3-5), Event-based queries (6-7),    Complex analytical queries (8-10)

\begin{enumerate}
    \item What are the most common musical instruments?
    \item Explain the concept of harmony in music
    \item How has music evolved over the centuries?
    \item What are the characteristics of music in the Renaissance period?
    \item When did electronic music emerge?
    \item What impact did the invention of recording technology have on music?
    \item How did jazz influence modern music genres?
    \item Compare Western and Eastern musical traditions
    \item What is the relationship between rhythm and cultural identity in music?
    \item How do musical scales affect emotional perception?
\end{enumerate}

For each of the 10 text querries, we first retrieve the top 5 relevant documents from the vector store. Then we use the 2 method above to evaluate the relevant score of the retrieved docs. 

\begin{enumerate}
    \item For method 1, we calculate the embedding of the query and of each document, then we calculate the cosine similarity between the embedding of query with the embedding of each doc.
    \item For method 2, we feed a prompt to the LLM to ask it to give a score based on how relevant the document is to the query. The model used is \texttt{gemini-2.5-flash}.
    
    \begin{lstlisting}[language=python, caption=LLM-as-a-judge]
for i, doc in enumerate(retrieved_docs):
    prompt = f"""Rate the relevance of the following document to the query on a scale of 1-5.

Query: {query}

Document Content: {doc.page_content[:500]}...

Scoring scale:
5 - Highly relevant, directly answers the query
4 - Relevant, provides useful information
3 - Somewhat relevant, tangentially related
2 - Minimally relevant, barely related
1 - Not relevant, unrelated to query

Respond with ONLY the numeric score (1-5)."""
    
    try:
        response = model.invoke(prompt)
        score = int(response.content.strip())
        scores.append(min(max(score, 1), 5))  # Ensure score is between 1-5
    except:
        scores.append(3)  # Default to neutral if parsing fails
    \end{lstlisting}
\end{enumerate}

Here is the result:

\begin{lstlisting}[caption={RAG score result}]
============================================================
Query 1: What are the most common musical instruments?
============================================================

Metrics:
  Retrieval time: 1.732s
  Avg similarity: 0.697
  Avg relevance: 2.60/5

Top 3 Retrieved Documents:

  Document 1:
    Similarity: 0.715
    Relevance: 3/5
    Source: https://en.wikipedia.org/wiki/Musical_instrument
    Content: as the only system that applies to any culture and, more importantly, provides the only possible classification for each instrument. The most common c...

  Document 2:
    Similarity: 0.703
    Relevance: 4/5
    Source: https://en.wikipedia.org/wiki/Percussion_instrument
    Content: === By prevalence in common knowledge ===
It is difficult to define what is common knowledge but there are instruments percussionists and composers us...

  Document 3:
    Similarity: 0.697
    Relevance: 2/5
    Source: https://en.wikipedia.org/wiki/Musical_instrument
    Content: There are many different methods of classifying musical instruments. Various methods examine aspects such as the physical properties of the instrument...

============================================================
Query 2: Explain the concept of harmony in music
============================================================

Metrics:
  Retrieval time: 0.483s
  Avg similarity: 0.755
  Avg relevance: 5.00/5

Top 3 Retrieved Documents:

  Document 1:
    Similarity: 0.773
    Relevance: 5/5
    Source: https://en.wikipedia.org/wiki/Music
    Content: Harmony refers to the "vertical" sounds of pitches in music, which means pitches that are played or sung together at the same time creates a chord. Us...

  Document 2:
    Similarity: 0.764
    Relevance: 5/5
    Source: https://en.wikipedia.org/wiki/Music_theory
    Content: In music, harmony is the use of simultaneous pitches (tones, notes), or chords. The study of harmony involves chords and their construction and chord ...

  Document 3:
    Similarity: 0.751
    Relevance: 5/5
    Source: https://en.wikipedia.org/wiki/Music
    Content: === Harmony ===...

============================================================
Query 3: How has music evolved over the centuries?
============================================================

Metrics:
  Retrieval time: 0.465s
  Avg similarity: 0.720
  Avg relevance: 4.20/5

Top 3 Retrieved Documents:

  Document 1:
    Similarity: 0.731
    Relevance: 5/5
    Source: https://en.wikipedia.org/wiki/Popular_music
    Content: There are multiple possible explanations for many of these changes. One reason for the brevity of songs in the past was the physical capability of rec...

  Document 2:
    Similarity: 0.730
    Relevance: 5/5
    Source: https://en.wikipedia.org/wiki/Classical_music
    Content: By the 20th century, stylistic unification gradually dissipated while the prominence of popular music greatly increased. Many composers actively avoid...

  Document 3:
    Similarity: 0.718
    Relevance: 3/5
    Source: https://en.wikipedia.org/wiki/Popular_music
    Content: In addition to many changes in specific sounds and technologies used, there has been a shift in the content and key elements of popular music since th...

============================================================
Query 4: What are the characteristics of music in the Renaissance period?
============================================================

Metrics:
  Retrieval time: 0.479s
  Avg similarity: 0.724
  Avg relevance: 4.40/5

Top 3 Retrieved Documents:

  Document 1:
    Similarity: 0.744
    Relevance: 5/5
    Source: https://en.wikipedia.org/wiki/Music
    Content: Renaissance music (c.1400 to 1600) was more focused on secular themes, such as courtly love. Around 1450, the printing press was invented, which made...

  Document 2:
    Similarity: 0.734
    Relevance: 5/5
    Source: https://en.wikipedia.org/wiki/Classical_music
    Content: ==== Renaissance ====

The musical Renaissance era lasted from 1400 to 1600. It was characterized by greater use of instrumentation, multiple interwea...

  Document 3:
    Similarity: 0.722
    Relevance: 4/5
    Source: https://en.wikipedia.org/wiki/Baroque_music
    Content: tritone, perceived as an unstable interval, to create dissonance (it was used in the dominant seventh chord and the diminished chord). An interest in ...

============================================================
Query 5: When did electronic music emerge?
============================================================

Metrics:
  Retrieval time: 0.571s
  Avg similarity: 0.728
  Avg relevance: 4.60/5

Top 3 Retrieved Documents:

  Document 1:
    Similarity: 0.741
    Relevance: 5/5
    Source: https://en.wikipedia.org/wiki/Electronic_music
    Content: The first electronic musical devices were developed at the end of the 19th century. During the 1920s and 1930s, some electronic instruments were intro...

  Document 2:
    Similarity: 0.740
    Relevance: 5/5
    Source: https://en.wikipedia.org/wiki/Electronic_music
    Content: During the 1960s, digital computer music was pioneered, innovation in live electronics took place, and Japanese electronic musical instruments began t...

  Document 3:
    Similarity: 0.724
    Relevance: 5/5
    Source: https://en.wikipedia.org/wiki/Electronic_music
    Content: === United States ===
In the United States, electronic music was being created as early as 1939, when John Cage published Imaginary Landscape, No. 1, ...

============================================================
Query 6: What impact did the invention of recording technology have on music?
============================================================

Metrics:
  Retrieval time: 0.472s
  Avg similarity: 0.708
  Avg relevance: 5.00/5

Top 3 Retrieved Documents:

  Document 1:
    Similarity: 0.715
    Relevance: 5/5
    Source: https://en.wikipedia.org/wiki/Music
    Content: distributed. The introduction of the multitrack recording system had a major influence on rock music, because it could do more than record a band's pe...

  Document 2:
    Similarity: 0.712
    Relevance: 5/5
    Source: https://en.wikipedia.org/wiki/Popular_music
    Content: In the 1950s and 1960s, the new invention of television began to play an increasingly important role in disseminating new popular music. Variety shows...

  Document 3:
    Similarity: 0.709
    Relevance: 5/5
    Source: https://en.wikipedia.org/wiki/Music
    Content: In the 19th century, a key way new compositions became known to the public was by the sales of sheet music, which middle class amateur music lovers wo...

============================================================
Query 7: How did jazz influence modern music genres?
============================================================

Metrics:
  Retrieval time: 0.539s
  Avg similarity: 0.705
  Avg relevance: 3.60/5

Top 3 Retrieved Documents:

  Document 1:
    Similarity: 0.719
    Relevance: 5/5
    Source: https://en.wikipedia.org/wiki/Rock_music
    Content: fusion began to take its audience, but acts like Steely Dan, Frank Zappa and Joni Mitchell recorded significant jazz-influenced albums in this period,...

  Document 2:
    Similarity: 0.712
    Relevance: 3/5
    Source: https://en.wikipedia.org/wiki/Music
    Content: Jazz evolved and became an important genre of music over the course of the 20th century, and during the second half, rock music did the same. Jazz is ...

  Document 3:
    Similarity: 0.702
    Relevance: 5/5
    Source: https://en.wikipedia.org/wiki/Rock_music
    Content: British acts to emerge in the same period from the blues scene, to make use of the tonal and improvisational aspects of jazz, included Nucleus and the...

============================================================
Query 8: Compare Western and Eastern musical traditions
============================================================

Metrics:
  Retrieval time: 0.499s
  Avg similarity: 0.720
  Avg relevance: 4.40/5

Top 3 Retrieved Documents:

  Document 1:
    Similarity: 0.734
    Relevance: 4/5
    Source: https://en.wikipedia.org/wiki/Music
    Content: In the West, much of the history of music that is taught deals with the Western civilization's art music, known as classical music. The history of mus...

  Document 2:
    Similarity: 0.720
    Relevance: 4/5
    Source: https://en.wikipedia.org/wiki/Classical_music
    Content: Classical music generally refers to the art music of the Western world, considered to be distinct from Western folk music or popular music traditions....

  Document 3:
    Similarity: 0.715
    Relevance: 4/5
    Source: https://en.wikipedia.org/wiki/Music
    Content: === Asian cultures ===

Asian music covers a swath of music cultures surveyed in the articles on Arabia, Central Asia, East Asia, South Asia, and Sout...

============================================================
Query 9: What is the relationship between rhythm and cultural identity in music?
============================================================

Metrics:
  Retrieval time: 0.477s
  Avg similarity: 0.696
  Avg relevance: 3.20/5

Top 3 Retrieved Documents:

  Document 1:
    Similarity: 0.709
    Relevance: 5/5
    Source: https://en.wikipedia.org/wiki/Tempo
    Content: song (although this would be less likely with an experienced bandleader). Differences in tempo and its interpretation can differ between cultures, as ...

  Document 2:
    Similarity: 0.698
    Relevance: 3/5
    Source: https://en.wikipedia.org/wiki/Tempo
    Content: This context-dependent perception of tempo and rhythm is explained by the principle of correlative perception, according to which data are perceived i...

  Document 3:
    Similarity: 0.695
    Relevance: 2/5
    Source: https://en.wikipedia.org/wiki/Music
    Content: === Rhythm ===...

============================================================
Query 10: How do musical scales affect emotional perception?
============================================================

Metrics:
  Retrieval time: 0.451s
  Avg similarity: 0.705
  Avg relevance: 3.00/5

Top 3 Retrieved Documents:

  Document 1:
    Similarity: 0.709
    Relevance: 5/5
    Source: https://en.wikipedia.org/wiki/Music_theory
    Content: The interrelationship of the keys most commonly used in Western tonal music is conveniently shown by the circle of fifths. Unique key signatures are a...

  Document 2:
    Similarity: 0.708
    Relevance: 4/5
    Source: https://en.wikipedia.org/wiki/Scale_(music)
    Content: Tetratonic (4 notes), tritonic (3 notes), and ditonic (2 notes): generally limited to prehistoric ("primitive") music
Scales may also be described by ...

  Document 3:
    Similarity: 0.704
    Relevance: 2/5
    Source: https://en.wikipedia.org/wiki/Soundtrack
    Content: plot anticipations, and moral judgement of the characters. Furthermore, eyetracking and pupillometry studies found that film music is able to influenc...

============================================================
\end{lstlisting}

Overall, we can see that the time it take to retrieve the context is acceptable (less than $1$ second for most cases). The average similarity score is around 0.7-0.75; and the average LLM judge rating varied more, around 2.6 to 5.0. LLM judge score can be pretty inconsistance and biased, or even inaccurate, so we use extra  manual inspection to check the relevance of the retrived context to the query; but generally, the LLM can be quite good at deciding which infomation is important to answer the query. 

In conclusion, we can say that with our vector database of 76 Wiki articles is quite sufficient for our RAG system, and RAG did a good job at retrieving the neccesary context for the queries.
\pagebreak
\section{Conclusion}

In this project, we have successfully designed and implemented an AI Agent chatbot specializing in the music domain, leveraging the power of LLM and RAG. By integrating LangChain, a vector database constructed from 76 Wikipedia articles, and speech interfaces (Text-to-Speech and Speech-to-Text), we created an interactive system capable of understanding and answering complex user queries with contextually relevant information.

Our theoretical exploration covered the fundamental building blocks of modern AI, including Artificial Neural Networks, Transformer architectures, and the evolution of LLMs. We delved into the specifics of RAG, examining advanced techniques such as reranking and reasoning to overcome the limitations of standard vector-based retrieval.

The evaluation of our system demonstrated the effectiveness of the RAG approach. The system achieved sub-second retrieval times and maintained consistent relevance scores, with cosine similarity averaging between 0.7 and 0.75. The "LLM-as-a-Judge" evaluation, complemented by manual inspection, confirmed that the retrieved context significantly enhanced the quality and accuracy of the generated responses compared to a standalone model.

Looking ahead, there are several avenues for future improvement. Expanding the knowledge base beyond the initial set of Wikipedia articles would broaden the agent's expertise. Implementing more sophisticated reranking algorithms and exploring reinforcement learning techniques could further refine retrieval accuracy. Additionally, optimizing the latency of the speech interfaces would contribute to a more seamless real-time user experience. Overall, this project validates the potential of domain-specific RAG agents in bridging the gap between general-purpose LLMs and specialized information needs.
\pagebreak

% REFERENCES
\pagebreak
\nocite{*}
\printbibliography[
heading=bibintoc,
title={References}
]
\end{document}

